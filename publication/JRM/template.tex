\documentclass[paper,JRM,paper]{jaciiiarticle}
\usepackage[dvips]{graphicx} 
\usepackage{epsfig}  
\usepackage{upgreek}
\usepackage{booktabs}
\usepackage{multirow}
\usepackage{siunitx}
\usepackage{amsmath}
\usepackage[normalem]{ulem}
\useunder{\uline}{\ul}{}


%%% following commands are unnecessary to edit %%%
\setcounter{page}{1}
\SetVolumeNumber{Vol.0 No.0, 200x} 
\SetArticleID{Rb*-**-**-****:}
%\SetIndexTab{generalarticle}%specialissue}%   review
%%%%%%%%%%%%%%%%%%%%%%%%%%%%%%%%%%%%%%%%%%%%%%%%%%

\begin{document}

\pagestyle{jaciiistyle}

\title{Active Passive Nature of Assistive Wearable Augmented Gait Suit for Enhanced Mobility}

\author{Chetan Thakur$^1$, Kazunori Ogawa$^1$$^,$$^2$, and Yuichi Kurita$^1$$^,$$^3$}
\address{$^1$Hiroshima University, Hiroshima, Japan\\
		 $^2$ Daiya Industries Co, Ltd. Okayama, Japan\\
		 $^3$ JST PRESTO, Saitama, Japan\\
         E-mail: chetanthakur@hiroshima-u.ac.jp}
\markboth{Thakur, C. et al.}{Assistive Wearable Gait Augment Suit (AWGAS) for Enhanced Mobility and Reduced Muscle Activation of Posterior and Anterior Muscles of Lower Limb.}
%\dates{00/00/00}{00/00/00}
\maketitle

\begin{abstract}% Do not delete this percent symbol
\noindent Abstract. %please write down your abstract here without first indent

In this paper we discuss the design, control and active passive nature of the assistive wearable gait augment suit (AWGAS).  AWGAS is a soft, wearable, lightweight and assist walking gait by reducing muscle activation during walking. It augments walking by reducing the muscle activation of the posterior and anterior muscles of the lower limb. The suit uses pneumatic gel muscles (PGM), foot sensors for gait detection, and pneumatic valves to control the air pressure. The assistive force is provided using the motion in loop feedforward control loop using  foot sensors in shoes. PGMs are actuated with the help of pneumatic valves and portable air tanks. The elastic nature of the PGM allows AWGAS to assist walking in the absence of the air supply which makes AWGAS both active and passive walking assist suit. To evaluate the active and passive nature of the AWGAS, we experimented to measure surface EMG (sEMG) of the lower limb muscles. sEMG was recorded for unassisted walking, i.e., without the suit, passive assisted walking, i.e., wearing the suit with no air supply and active assisted walking, i.e. wearing the suit with air supply set at 60 kPa.  The results show a reduction in the muscle activity for both passive and active assisted walking as compared to unassisted walking. The pilot trials of the AWGAS were conducted in collaboration with local farmers in the Hiroshima prefecture in Japan where feedback received is complementing the results obtained during the experiments.

\end{abstract}

\begin{keywords}
wearable assistive suit, artificial muscle, gait, pneumatic gel muscle, unplugged powered suit, human augmentation, walking assist, motion in loop control
\end{keywords}


\section{Introduction}

Exoskeletons and exosuits augment human capability and reduce the chance of muscle fatigue, accidents or musculoskeletal disorders in the long term. These devices are used in settings such as factories where work involves lifting heavy objects, agriculture, walking long distances or sports. The objective of exoskeletons is to reduce human muscle efforts and augment the human ability to perform various tasks that could not be possible otherwise. Whereas, the objective of exosuits is to augment human motion and provide assistive force, thereby reducing muscle effort. With the growing elderly population, stressful work conditions, and eagerness to live a good quality of life, such devices will be critical. 

Among various exoskeletons, HAL \cite{1} makes walking easier and useful for the rehabilitation process. Wearable agri robot \cite{2} designed to reduce physical strain on the body during farming activity. Walking assist device \cite{3} reduces floor reaction force of the user and assist walking, stair climbing and squats motion. A passive plantarflexion assist exoskeleton \cite{4} reduces the metabolic cost of walking. In stand-alone wearable power assist suits for caregivers, using pneumatic actuators \cite{5}, the suit provides extra strength and reduces the muscle effort required for the desired task up to 50\%. Exoskeleton to support knee joint motion \cite{6} uses pneumatic artificial muscles to reduce the stress on the lower back and knees during the performance of loading and unloading tasks for manual workers. Wearable assistive suits for walking and lifting-up \cite{7}, with 13 DOF, and reduces muscle efforts significantly. A power assist device for standing up motion \cite{8} for elderly or disabled to lead self-supporting lives. \cite{9,10,11,12} are some of the wearable exosuits to assist walking and reduces the metabolic cost of walking using their unique methods. 

These exoskeletons and exosuits are classified in various segments such as healthcare, disability support, industrial and augmented motion. They fulfill their respective purposes effectively using their unique methods, but the application outside the controlled environment is limited and not discussed especially for agriculture or outdoor environments. For augmenting human gait in an outdoor environment using exoskeleton or exosuits, wearable, portable, easy to use and light weight are some of the essential aspects to be considered, but together they are missing in the devices discussed earlier. These devices require large batteries or compressor or large air tanks for actuation of pneumatic artificial muscles (PAM), which makes device heavy and difficult to use. Most of these devices lack augmentation capability in the absence of the power sources such as batteries needed for actuator control or compressed air tanks in case of assistive suit using PAM. For the outdoor environment, it is essential that the device possess both active and passive nature to provide no disturbance or provide minimal assistance for human motion in the absence of power source. In our previous research \cite{13}, we developed a low-powered PAM, termed as pneumatic gel muscle (PGM). The PGM has an inner tube constructed with the styrene-based thermoplastic elastomer and outer protective mesh. This artificial muscle can be actuated with 50 kPa using small rubber pump and possess elastic nature to provide the minimal assistive force in the absence of air pressure. Such actuation capacity is missing in the tradition PAM used for developing wearable assistive suits. In \cite{13} K Ogawa et al. also devised a mechanism and developed an unplugged powered suit (UPS) to augment human walking.

Here, we discuss the enhanced design of the UPS using the motion in the loop feed-forward control algorithm for generating assistive force to improve the adverse effect on the antagonistic muscles discussed in \cite{13}. In section 2 we will discuss the drawbacks of PGM and UPS, AWGAS design, biomechanics of the gait cycle, control algorithm and modeling of the assistive force generated by the PGM. In section 3, we discuss the evaluation and results of the AWGAS experiment. In section 4, we discuss preliminary trials in the rural area of the Hiroshima prefecture followed by discussion, conclusion to discuss the passive and active nature of the suit based on reduced muscle activity in the lower limb muscles.




\section{Background and Methodology}

\subsection{Pneumatic Gel Muscle (PGM) and Unplugged Powered Suit (UPS)}
In this section, we discuss our research background. In our previous research\cite{13}, we developed a low-pressure-driven artificial muscle, PGM. Unlike traditional PAMs \cite{14}, the PGM is driven by low air pressure. K. Ogawa et al. reported that the PGM possesses the force-generating capacity from \SI{50}{\kilo\pascal} to \SI{300}{\kilo\pascal} of air pressure. Fig. \ref{fig:pgmreal} shows the schematics and image of the actual product; it has a natural length of \SI{300}{\milli\meter}, the maximum possible contraction is \SI{200}{\milli\meter}, the maximum elongation is \SI{500}{\milli\meter}, and the maximum possible force is \SI{50}{\newton}. Based on the PGM, we also designed and prototyped the UPS. The UPS requires no electricity but uses the walking motion to power the PGM with the help of the pump in the shoe. The evaluation of the UPS shows that it reduces the muscle efforts in the swing phase of the gait cycle; however, the effect is invisible for all the muscles of the lower limb owing to a small amount of assistive force generated. For better performance, the use of external power sources such as an air tank and a pneumatic valve control can enhance the augmentation factor of the assistive suit. 

\begin{figure}
	\centering
	\includegraphics[width=0.9\linewidth]{photos/pgmreal}
	\caption{Pneumatic Gel Muscle and illustration of force-generating mechanism}
	\label{fig:pgmreal}
\end{figure}


The experiment was conducted to examine and distinguish the PGM actuation delay for the pump and the air tank as a source of air pressure, by recording the air pressure, air flow rate, and force-generation with respect to time. The objective of this experiment was to identify the difference in the time when the air pressure is supplied to when the PGM generates the force. Fig. \ref{fig:activationdelayexperiment} shows the experimental setup; it consists of a PGM with both ends hooked up with one end attached to the Phidgets loadcell to record generated force, the air pressure sensor at the inlet valve of the PGM, and the air flow sensor to record the air flow rate at various air pressures. Table \ref{sensordetails} lists the specifications of the sensor. In this experiment, we used a rubber pump and a portable air tank to actuate the PGM. The delay was recorded for two PGM configurations, first when the PGM is attached at its rest length and second when the PGM is stretched to  \SI{370}{\milli\meter}. The graphs in Fig. \ref{fig:pump} and Fig. \ref{fig:tank} show the actuation delay profiles of the PGM for the air sources of pump and air tank. We observed that when the pump is used, the delay in the generated force is approximately \SI{260}{\milli\second} and \SI{130}{\milli\second} for the rest length of \SI{300}{\milli\meter} and the stretched length of \SI{370}{\milli\meter}, respectively. However, when the air tank is used as a source, no delay is observed between the supplied air pressure and generated force.

\begin{figure}
	\centering
	\includegraphics[width=0.9\linewidth]{photos/activationdelayexperiment}
	\caption{Experiment setup to measure actuation delay of rubber pump and portable air tank used as air pressure source.}
	\label{fig:activationdelayexperiment}
\end{figure}

% Please add the following required packages to your document preamble:
% \usepackage{booktabs}
\begin{table}[]
	\caption{Specifications of sensors used in the experiment}
	\begin{tabular}{@{}lll@{}}
		\toprule
		\multicolumn{1}{c}{Category} & \multicolumn{1}{c}{Sensor Name} & \multicolumn{1}{c}{Specification} \\ \midrule
		Air Flow Rate Sensor & PFMV-530-1  & 0-+3.0 L/min \\
		Air Pressure Sensor & PSE540-1M5  & 0-1Mpa \\
		Loadcell force sensor & CZL653   & 0-200N \\
		PhidgetBridge & 1046\_0B  &  \\ \bottomrule
	\end{tabular}

	\label{sensordetails}
\end{table}

\begin{figure}[h]
	\centering
	\includegraphics[width=1\linewidth]{photos/pumpdelay}
	\caption{Actuation delay caused when the pump is used as the air pressure source. The first row shows the generated air pressure and the second row shows the generated force with respect to time.}
	\label{fig:pump}
\end{figure}

\begin{figure}
	\centering
	\includegraphics[width=1\linewidth]{photos/tankdelay}
	\caption{Actuation delay caused when the tank is used as the air pressure source. The first row shows the generated air pressure and the second row shows the generated force with respect to time.}
	\label{fig:tank}
\end{figure}

Based on this observation, we found that for the UPS, generating the assistive force precisely is difficult owing to the delay and the difficulty in attaching multiple pumps in the shoe, which also disturbs normal walking. Therefore, we developed a PGM actuation control based on the gait detection system that can benefit and elevate the performance of the walking assist suit. In the next section, we discuss the design and control of a newly developed walking assist suit.

\subsection{Design of Assistive Wearable Gait Augment Suit (AWGAS)}
To overcome the challenge of the UPS, we developed a new walking assist suit called the AWGAS. AWGAS is developed based on the principle of the UPS, i.e., a lightweight, wearable, low-powered assistive suit. For the AWGAS, pneumatic solenoid valves are used for the actuation control of the PGM to generate the assistive force, and the control signal was derived based on the gait detection system developed using pressure sensors in the shoes. The new design enables it to support a walking pitch of more than two steps per second which was not possible in UPS \cite{13} and the level of augmentation can be controlled by setting the maximum supply air pressure.  K. Ogawa et al. mentioned that the maximum air pressure that the PGM can handle is 300 kPa, and contraction ratio increases with the air pressure. Therefore, setting the maximum cutoff air pressure using a pressure regulator decides the augmentation factor of the suit during walking. 

The AWGAS consists of a waist support, a knee support, PGMs attached along the quadriceps femoris, pressure sensors, pneumatic solenoid valves, controllers, and a portable air tank. Fig. \ref{fig:aws} and Fig. \ref{fig:awgasillustration} show the overview and illustration of the applied assistive force during the swing phase of the gait cycle. Fig.\ref{fig:aws} shows that the PGM is attached along the thigh muscles with the help of the waist support and knee support, while the backpack contains the controller circuit, the portable air tank, and a battery for the controller, and the pressure sensors in the shoe are connected to the controller. Fig. \ref{fig:awgasillustration} illustrates the applied assistive force; the elastic nature of the PGM allows it to stretch during the stance and terminal stance phases and from the terminal stance controller that actuates the PGM for the swing phase of the gait cycle.

Table \ref{awgasconfig} lists all the components of the AWGAS with respective specification and weights. Based on this total weight of AWGAS is \SI{2}{\kilogram} approximately , the battery lasts for 3-4 days of runtime and the air tank can actuate 500 actuation at \SI{100}{\kilo\pascal} when tested in the lab. 

\begin{figure}[h]
	\centering
	\includegraphics[width=1\linewidth]{photos/awgas_sue}
	\caption{AWGAS Design and Construction}
	\label{fig:aws}
\end{figure}
\begin{figure}[h]
	\centering
	\includegraphics[width=1\linewidth]{photos/AWGASillustration}
	\caption{Applied assistive force during swing phase of the gait cycle supporting forward locomotion}
	\label{fig:awgasillustration}
\end{figure}


The AWGAS uses the PGMs along the quadriceps femoris muscles connected across two joints: the pelvis and the knee. This configuration provides the assistive force during the swing phase of the gait cycle, and reduces the muscle effort of the quadriceps femoris during the swing phase of the gait cycle. The air pressure supply is controlled using the pneumatic solenoid valve. We also developed a gait detection system that identifies the swing phase of the individual limb. This system consists of two FSR-406 pressure sensors placed in each shoe. These are used together to detect the walking motion, standing posture, and gait cycle of the individual limb, and identify the phase of the gait cycle.

% Please add the following required packages to your document preamble:
% \usepackage{booktabs}
\begin{table}[]
	\centering
	\caption{Components of the AWGAS, their specification and weights in kilogram.}
	\begin{tabular}{@{}cccc@{}}
		\toprule
		\textbf{Name} & \textbf{Quantity} & \textbf{Specification}                                             & \textbf{Weight (kg)} \\ \midrule
		Air tank      & 1                 & 40 (liters)                                                        & 0.3                  \\
		Regulator     & 1                 & 0 - 0.5 Mpa)                                                       & 0.34                 \\
		PGM           & 2                 & 30 cm                                                              & 0.11                 \\
		Tubes         & 1                 & \begin{tabular}[c]{@{}c@{}}Polyurethane \\ (4 mm)\end{tabular}     & 0.022                \\
		Knee Support  & 2                 & Fabric                                                             & 0.052                \\
		Waist Support & 1                 & Fabric                                                             & 0.11                 \\
		Controller    & 1                 & Arduino Uno                                                        & 0.3                  \\
		Battery       & 1                 & \begin{tabular}[c]{@{}c@{}}10000 mAh \\ / 38.48 watts\end{tabular} & 0.24                 \\
		Backpack      & 1                 &                                                                    & 5                    \\ \midrule
		\multicolumn{3}{c}{Total Weight of AWGAS}                                                              & 1.974                \\ \bottomrule
	\end{tabular}

	\label{awgasconfig}
\end{table}

%Components of the AWGAS, their specification and weights in kilogram

\subsection{Biomechanics of Gait Cycle and Control strategy of AWGAS}

Human walking is synchronized motion of both limbs called as gait. A gait cycle also called as stride is motion of a limb from stance to stance phase while walking. Fig. \ref{fig:gaitcycle} illustrates the gate cycle as described in \cite{15}, its phases, location, and orientation of foot and region of assisted motion. In the gait cycle, the initial 10\% of each phase is called the double limb support. In this part of the gait cycle, the limb transfers from stance to swing, and vice versa, on a contralateral foot. The stance phase is responsible for load balance and maintain posture whereas swing phase mainly responsible for forward movement. During this knee extensors and flexors are used from toe off to heel strike. Assisting this phase, we can reduce the effort required by these muscle group thereby augmenting gait. 

To reduce the muscle efforts during swing phase of the gait cycle we need to identify the beginning and end of the gait cycle. In our design this was done by using force sensitive resistors (FSR) in the shoe. The FSRs are places at the heel of the shoe. The placement of the FSR is used for detecting stance phase of the gait cycle. Since the gait is synchronous activity of both limb when one limb is in the stance phase other limb is in the swing phase. The dual support phase during which transition between stance and swing happens, helps us identify the limb in swing phase using FSR sensors in the shoe. Based on this the system identifies which limb to assist and when to start and stop assist i.e. actuate PGM using pneumatic solenoid valves. CHEN  et al. in \cite{16} reported that the time required for a single limb support is equal to the swing time and can be calculated based on the FSR or inertial sensors. Similarly, the use of foot sensors is very common for exoskeletons and assistive suits \cite{16,17,18,19,20,21}, and provides accuracy in gait cycle detection. CHETN et al. also mentioned the time required for the stance phase of one limb is equal to time required for swing phase of the contralateral limb.

Based on this we designed the algorithm for actuation control of PGM actuators during swing phase and synchronize the assistive force with the gait cycle. Fig. \ref{fig:gaitdetectflowchart} shows the flowchart of the actuation control logic for AWGAS. In the flowchart FSR sensors in both limbs are used to detect gait cycle i.e. to identify if the user is walking. Upon detecting the gait cycle the algorithm identifies the limb in the swing phase and actuates PGM actuator to provide the assistive force by turning on pneumatic solenoid valve. This algorithm is continuous process of polling FSR sensor readings and deciding the actuation control. 


\begin{figure}[h]
	\centering
	\includegraphics[width=1\linewidth]{photos/gaitcycle}
	\caption{Classification of the gait cycle into stance and swing phases and the respective orientation of the foot. Region shows where assistive force is provided. (DLS = Double limb support \& SLS = Single limb support) }
	\label{fig:gaitcycle}
\end{figure}

\begin{figure}[h]
	\centering
	\includegraphics[width=0.7\linewidth]{photos/gaitdetectflowchart}
	\caption{Flow chart of PGM actuation logic for generating assistive force}
	\label{fig:gaitdetectflowchart}
\end{figure}

\begin{figure}
	\centering
	\includegraphics[width=1\linewidth]{photos/assistcontrol}
	\caption{Functional operation of AWGAS actuation control using gait detection system. The upper graph is for gait detection. The graph below shows the generation of assistive force for the swing phase duration of the left limb.}
	\label{fig:assistcontrol}
\end{figure}

\begin{align}\label{eq:actu1}
f(PGM_{actuation}) = f(R,L)
\end{align}
where
$f(PGM_{actuation})$ is actuation control function which depends on $R = f(R_{fsr})$ and $L = f(L_{fsr})$ which are functions of FSR sensor in right and left limb respectively. 
$FSR_{ref}$ is reference value of FSR sensor used for distinguishing between stance and swing phase while walking. 	

$R$ and $L$ are defined as as follows

\begin{align}\label{eq:rfsr}
R= ^{}
\begin{cases}
1,& \text{if } f(R_{fsr)} \geq FSR_{ref}, \text{  i.e. Stance Phase}\\
0              & \text{otherwise, i.e. Swing Phase}
\end{cases}
\end{align}

\begin{align}\label{eq:lfsr}
L= ^{}
\begin{cases}
1,& \text{if } f(L_{fsr)} \geq FSR_{ref}, \text{  i.e. Stance Phase}\\
0,              & \text{otherwise, i.e. Swing Phase}
\end{cases}
\end{align}

From Eq \ref{eq:rfsr} and Eq \ref{eq:lfsr} we can clearly identify the present phase of the respective limb. Based on this we update Eq \ref{eq:actu1} as follows:

\begin{align}\label{eq:actufinal}
f(PGM_{actuation}) = ^{}
\begin{cases}
L_{assist},& \text{ if } R = 1\text{ \& } L = 0\\ 
R_{assist},& \text{ if } R = 0\text{ \& } L = 1\\ 
%\text{No Assist}, & \text{ if } (R \text{ \& } L = 1) OR (R \text{ \& } L = 0)
\text{No Assist}, & \text{otherwise}
\end{cases}
\end{align}

Eq. \ref{eq:actufinal} states the developed actuation control of the PGM based on antagonistic stance phase detection and synchronizes with the normal gait cycle. Fig. \ref{fig:assistcontrol} shows the gait cycle information with reference to the FSR sensors in the shoe and the generation of assistive force. It shows the simultaneous stance phase detection of the right and left limb and the generation of an assistive force for the left limb, which is in the swing phase of the gait cycle. The intensity of assistive force depends on the supplied air pressure which is decided by the user based on individual need for augmentation. This designed assistive force is ideal for assisting forward motion of the limb in the swing phase of the normal gait. The assistance during swing phase was designed to reduce muscle efforts for quadriceps, hamstring and planter flexor muscles by using one PGM along quadriceps (RF) muscle. 


\subsection{Modeling of assistive forces}
In the previous section, we discussed the importance of phases in a gait cycle and how it was utilized to develop assist and actuation control of PGM synchronized with gait cycle. In this section we will discuss the modeling of force generation characteristics of PGM under the influence of tension in the muscle due to stretched use. 

K. Ogawa et al. in \cite{13} measured the force and displacement characteristics for the change in the internal air pressure. These characteristics describe the displacement due to elongation when the load is attached and changed in air pressure. They reported that the elongation ratio changes as the supply air pressure is changed for the same load. This observation states the load carrying capacity of PGM and elongation for the change in air pressure. But does not consider the tension in the muscle i.e. when both ends of muscle is fixed and PGM is stretched before actuation. It is important to measure this characteristic because in the wearable assistive suits the PGM actuators will always have their both ends fixed at rest length or stretched depending on the application. To model this characteristic, we perform experiment to measure force generated due to stretch in the PGM. Experiment setup is shown in Fig. \ref{fig:activationdelayexperiment}, in this the force generated by PGM for tension in the muscle due to stretch was observed. The generated force was measured for rest length i.e. 30 cm and for stretched length of 32 cm, 34 cm and 37 cm. Fig. \ref{fig:tensionmodel}  shows the observed force profile of PGM due to tension caused by stretch. It shows the force increases linearly as we change the tension in muscle and air pressure. 



\begin{figure}[h]
	\centering
	\includegraphics[width=1\linewidth]{photos/pgmtensionmodel}
	\caption{Modeling of PGM force characteristics considering tension in the muscle due to stretched length and change in the air pressure. X-axis is supplied air pressure, Y-axis is force generated by PGM. The graph shows, force generated by PGM due to tension caused by stretching it. The solid lines are experiment values and dotted lines are modeled values.}
	\label{fig:tensionmodel}
\end{figure}

We modeled this observation as linear polynomial equation of $2^{nd}$ order of supplied air pressure and $1^{st}$ order of stretched length. Eq. \ref{eq:fpgm} defines this model where the coefficients are within the 95\% confidence bounds and $R^{2}$ value is 0.9852. In this linear model $x$ can be derived from stretched length of PGM when used in AWGAS and $y$ is the internal air pressure. The stretched length is the length PGM after it is stretched to attach along rectus femoris i.e. from pelvis to knee. Internal air pressure depends on the supply air pressure from the portable air tank using regulator. 

\[F_{pgm} = f(x,y)\]

where
\[x = \text{streched length of PGM}\] 
\[y = \text{supplied air pressue}\]

\begin{align}\label{eq:fpgm}
	F_{pgm} = a +bx+cy+dx^2+exy
\end{align}
where
\[a = -343.5\]
\[b=19.84\]
\[c = -0.6416\]
\[d = -0.2829\]
\[e = 0.02759\]

From this experiment and Eq.\ref{eq:fpgm} we observed, due to the elastic nature of the PGM, it can generate assistive force even in the absence of air pressure when stretched. Therefore, we believe the AWGAS developed using PGM should assist walking in the absence and presence of assistive air pressure.



\section{AWGAS Evaluation Experiment with sEMG and Variable Assistive Force.}

Gait cycle involves a complex sequence of the lower limb muscle activation together with joint moments at the hip, knee, and ankle. Augmenting the gait cycle involves reducing the muscle activation without affecting the walking speed or disturbing the motion. The AWGAS was developed to reduce muscle activation during the swing phase of the gait cycle by providing an assistive force using the PGM. The assistive force provided is based on the supplied air pressure. As discussed in the previous section, the AWGAS provides the assistive force for walking in the absence of air pressure. Based on this assumption, we evaluated the AWGAS for two conditions, one without air supply and the other with actuation control enabled and supply of air pressure. For the second case, the maximum possible air pressure was set to \SI{80}{\kilo\pascal}.

\subsection{Test Selection and Evaluation Criteria}
To evaluate the changes in the lower limb muscle activation for the use of AWGAS, surface EMG (sEMG) of eight significant muscles involved in the gait cycle and superficially accessible to record the sEMG were evaluated. We recorded muscle activities of the biceps femoris (BF) of hamstring muscles, quadriceps femoris muscles i.e. rectus femoris (RF), vastus medialis (VM) and vastus lateralis (VL), the calf muscles such as soleus (SOL), lateral gastrocnemius (LG) and medial gastrocnemius (MG), and tibialis anterior (TA) muscle. These muscles are significant contributors to the gait cycle and commonly used for assessing the gait experiments. 

For statistical analysis and comparison, multiple trials of the gait experiment with similar walking conditions were conducted, with each experiment measuring at least three full gait cycles excluding the beginning and the terminal gait cycle due to loading and unloading effect \cite{22}. In our study, we conducted three trials of each experiment and recorded ten full gait cycles. 

The experiment involves walking 15 m straight on a flat surface during the experiment sEMG, and the foot sensor data are recorded. Walking for 15 m provides ten full gait cycles, excluding the beginning and the terminal gait.  For recording the sEMG, we used the P-EMG plus device with eight channels with the recording frequency of \SI{1}{\kilo\hertz} for all the channels. The sEMG data are synchronized with the help of the FSR sensor data, which is beneficial for segmenting the gait cycles for the analysis and identification of the assist period in the gait cycle. 

Eight healthy adults seven male and one female (25.5 $\pm$ 4.8; mean $\pm$ s.d.) participated in the experiment. The details of the experiment were shared with all the participants before the experiment, and informed consent was acquired. During the experiment, the participants could rest anytime to avoid muscle fatigue during training and experiment session. Before starting the experiment, the maximum voluntary contraction (MVC) of each muscle was recorded. MVC is used for normalization of sEMG data for statistical analysis. To record the MVC, specific exercises or motions were performed by participants. They performed calf raises for recording the MVC for the calf muscles, i.e., SOL, LG, and MG, dorsiflexion for TA and squats for BF, and thigh contraction for VM, VL, and RF. 
After measuring MVC, participants were trained to maintain walking speed and get used to walking with the assistive suit. First, each subject walked 15 m repeatedly without an assistive suit and trained to maintain the walking speed. During this training, participants got familiar with the walking speed and surroundings. After this, participants wear AWGAS and practice walking to get familiar to walk with AWGAS. During which participants trained for assistive walking by wearing AWGAS with and without assistive air pressure provided. For each experiment the participants trained for approximately 20 mins. 



Fig. \ref{fig:experiment} shows a participant wearing the experiment setup. The participant wears the AWGAS, with the sEMG electrodes attached to respective muscles, shoes with the FSRs, a backpack, a P-EMG device for recording the sEMG, a laptop for connecting and operating the P-EMG application, a battery for powering the P-EMG device and AWGAS controller, a controller and actuation circuit for the AWGAS, and a portable air tank. Together with all the equipment, the weight of the backpack is 7 kg. The laptop in the backpack was remotely operated for recording the sEMG appropriately.


\begin{figure}[h]
	\centering
	\includegraphics[width=1\linewidth]{photos/experiment_sue}
	\caption{Subject wearing AWGAS with experiment setup}
	\label{fig:experiment}
\end{figure}

\subsection{Results}
The recorded sEMG data were rectified with an integrated EMG (iEMG), a second-order low-pass filter with 100 Hz cutoff frequency, and a second-order high-pass filter with cutoff frequency of 40 Hz. For statistical analysis and comparison with assisted motion and unassisted motion, the processed sEMG data were normalized as a percentage of the maximum voluntary contraction for an individual subject and their respective muscles. After normalizing the sEMG, ten full gait cycles were segmented and averaged with the help of the FSR sensor data. The FSR sensor was used to identify the gait cycle from heel strike to heel strike. Segmenting all the gait cycles yields one averaged gait cycle with its standard deviation. This was performed for the sEMG of all eight muscles for all the participants. The normalized and averaged sEMG of all participants were further averaged to prepare the gait cycle of the unassisted and assisted walking experiment. The final sEMG of all muscles were normalized and segmented, and the gait cycles for all participants were averaged with their standard deviations. Fig. \ref{fig:emgenvelope} and Fig. \ref{fig:semgpeakcompare} shows the sEMG envelope of the averaged and normalized gait cycle for all muscles with the respective standard deviation and compares the sEMG signal of all three experiments respectively. These figures shows that assisting swing phase of the gait cycle reduces the muscle effort during walking, based on the reduction in peak value of sEMG signal and less deviation in sEMG when walking with AWGAS.

For statistical analysis, we measured the average and standard deviation of the sEMG signal of each muscle for the assisted and unassisted gait. Fig. \ref{fig:awgasemgbarpvalue} shows a comparison of the average percentage MVC of each muscle for the unassisted and assisted gait experiment with their standard deviations. We conducted the student’s t-test assuming unequal variance to obtain the statistical significance of the reduction in the sEMG in the assisted gait. The test shows a significant difference for most of the muscles especially the muscles involved in the swing phase of the gait cycle for the AWGAS with no assist, and with actuation control and assist. The rectus femoris (p-value = 0.01 \& p-value = 0.011), biceps femoris (p-value = 0.004 \& p-value = 0.007), vastus medialis (p-value = 0.016 \& p-value = 0.002), vastus lateralis (p-value = 0.059 \& p-value = 0.019) and tibialis anterior  (p-value = 0.002 \& p-value = 0.001) are the active muscles during the swing phase of the gait cycle and show significant reduction while using the AWGAS without assist and with assist for the force generated at \SI{80}{\kilo\pascal}. The soleus (p-value = 0.56 \& p-value = 0.17), lateral gastrocnemius (p-value = 0.1 \& p-value = 0.14) and medial gastrocnemius (p-value = 0.024 \& p-value = 0.097) are the additional muscles responsible for the stance phase along with TA, RF, VM, and VL, which are also active during the swing phase. Based on the statistically significant difference discussed above, we observed that the assistive force provided during the swing phase of the walking reduced the muscle effort of the lower limb muscles in both the swing stance phases of the gait cycle. Meanwhile, the difference was more significant for the muscles involved in the swing phase, whereas for the other muscles in the stance phase, a reduction in the sEMG was observed, but this reduction was less significant. Table \ref{percentagereduction} shows the reduction in the \%MVC for both assisted conditions. The results of the t-test with the respective p-value and t-value are shown in table \ref{ttestresult}. 

Fig. \ref{fig:kinematics} shows kinematics changes in gait measured for active and passive assist when wearing assistive suit and without assistive suit. These changes are measured by onboard IMU sensor on Intel cuire based development board. From these results, the ankle trajectory is slightly shifted towards dorsiflexion and difference in dorsiflexion angle in assist condition is upto \ang{5}. No change in trajectory for knee and hip is observed whereas about \ang{5} reduction in knee extension is observed during swing phase of the gait cycle. 


Therefore, based on the graph in Figure \ref{fig:emgenvelope}, \ref{fig:awgasemgbarpvalue} and table \ref{ttestresult}, we observed that the use of AWGAS could reduce the muscle efforts during the swing phase and stance phase of the gait cycle significantly. We also observed that the AWGAS can assist walking when air pressure is supplied as well as in the absence of air pressure. 



\begin{figure*}
	\centering
	\includegraphics[width=0.7\linewidth]{photos/emgenvelope}
	\caption{Changes in the msuclemuscle activation pattern of the gait cycle due to the use of AWGAS with standard deviation. The X-axis signifies the gait cycle from 0 to 100 \% and the Y-axis signifies the average \%MVC sEMG activation pattern for each muscle under observation}
	\label{fig:emgenvelope}
\end{figure*}
\begin{figure*}
	\centering
	\includegraphics[width=1\linewidth]{photos/emgbar}
%	\includegraphics[width=1\linewidth]{photos/emgbar}
	\caption{Statistical evaluation of AWGAS based on the average of \% MVC for all subjectparticipants. X-axis is for all the muscles under observation and Y-axis is the average \% MVC for all subjectparticipants}
	\label{fig:awgasemgbarpvalue}
\end{figure*}
\begin{figure}
	\centering
	\includegraphics[width=1\linewidth]{photos/sEMG_peak_compare}
	\caption{Average normalized sEMG signal of muscles of right lower limb. X-axis is percentage of the gait cycle, Y-axis is average \%MVC for each muscle. This figure is useful for visual comparison of change in the sEMG signal.}
	\label{fig:semgpeakcompare}
\end{figure}

% Please add the following required packages to your document preamble:
% \usepackage{booktabs}
\begin{table}[]
	\centering
	\caption{Reduction average of \% MVC compared with normal gait. $-ve$ value indicates the increased in \% MVC, $+ve$ value represents the decrease in \% MVC.}
	\begin{tabular}{@{}ccc@{}}
		\toprule
		\textbf{Muscle} & \textbf{\begin{tabular}[c]{@{}c@{}}AWGAS no assist\\ \% of Normal Gait\end{tabular}} & \textbf{\begin{tabular}[c]{@{}c@{}}AWGAS assist\\ \% of Normal Gait\end{tabular}} \\ \midrule
		TA              & 14.73                                                                                & 17.55                                                                             \\
		SOL             & -8.69                                                                                & 6.76                                                                              \\
		MG              & 13.11                                                                                & 6.80                                                                              \\
		LG              & 22.39                                                                                & 8.87                                                                              \\
		RF              & 38.41                                                                                & 33.15                                                                             \\
		VM              & 7.65                                                                                 & 13.13                                                                             \\
		VL              & 21.77                                                                                & 20.18                                                                             \\
		BF              & 34.00                                                                                & 31.04                                                                             \\ \bottomrule
	\end{tabular}

	\label{percentagereduction}
\end{table}


% Please add the following required packages to your document preamble:
% \usepackage{booktabs}
% \usepackage{multirow}
\begin{table}[h]
	\centering
	\caption{Result of the Student's t-test with $p$ and $t$ values and $t-critical$ two-tail for testing our null hypothesis.}
	\resizebox{\linewidth}{!}{
	\begin{tabular}{@{}llccc@{}}
		\toprule
\textbf{Muscles}     & \multicolumn{1}{c}{\textbf{Experiment}} & \textbf{p-value} & \textbf{t-value} & \textbf{\begin{tabular}[c]{@{}c@{}}t Critical \\ two-tail\end{tabular}} \\ \midrule
		\multirow{2}{*}{TA}  & AWGAS No Assist                & 0.0022  & 14.89   & 4.30                                                           \\ 
		& AWGAS assist                   & 0.0019  & 16.35   & 4.30                                                           \\ \midrule
		\multirow{2}{*}{SOL} & AWGAS No Assist                & 0.0566  & -2.71   & 12.71                                                          \\ 
		& AWGAS assist                   & 0.1796  & 1.58    & 12.71                                                          \\ \midrule
		\multirow{2}{*}{MG}  & AWGAS no Assist                & 0.0244  & 4.36    & 4.30                                                           \\ 
		& AWGAS assist                   & 0.0973  & 3.17    & 12.71                                                          \\ \midrule
		\multirow{2}{*}{LG}  & AWGAS no Assist                & 0.1053  & 2.91    & 12.71                                                          \\ 
		& AWGAS assist                   & 0.1498  & 1.39    & 4.30                                                           \\ \midrule
		\multirow{2}{*}{RF}  & AWGAS no Assist                & 0.0102  & 31.18   & 12.71                                                          \\ 
		& AWGAS assist                   & 0.0111  & 28.62   & 12.71                                                          \\ \midrule
		\multirow{2}{*}{VM}  & AWGAS no Assist                & 0.0170  & 5.29    & 4.30                                                           \\ 
		& AWGAS assist                   & 0.0029  & 13.06   & 4.30                                                           \\ \midrule
		\multirow{2}{*}{VL}  & AWGAS no Assist                & 0.0595  & 5.29    & 12.71                                                          \\ 
		& AWGAS assist                   & 0.0200  & 15.93   & 12.71                                                          \\ \midrule
		\multirow{2}{*}{BF}  & AWGAS no Assist                & 0.0049  & 10.06   & 4.30                                                           \\ 
		& AWGAS assist                   & 0.0077  & 7.97    & 4.30                                                           \\ \bottomrule
	\end{tabular}
	}

	\label{ttestresult}
\end{table}

\begin{figure}
	\centering
	\includegraphics[width=0.9\linewidth]{photos/kinematics}
	\caption{Effect of AWGAS on gait kinematics. }
	\label{fig:kinematics}
\end{figure}


\section{Preliminary Trials and Assessment by Elderly Farmer}

The AWGAS is developed to augment walking and to be used by people of all age groups. To study the feasibility and usefulness of the AWGAS, we conducted the preliminary trials, where we shared the details of the AWGAS and its purpose with people in rural areas and requested to test the AWGAS and provide feedback. For successful participation, informed consent was obtained from all participants and pertinent information such as AWGAS specification and documentation, purpose of the trial, procedure, benefit and risks was translated in Japanese language and shared 4 months before the trials. 

During the trials, participants walked with and without AWGAS on paddy field while performing various daily rural tasks. They walked on a flat surface, up the slope, down the slope, and on uneven surfaces with and without the AWGAS. Fig. \ref{fig:preliminarystudy} shows the participant testing the AWGAS. After the trial, participants said they feel less effort required in the lower limb when using assistive suit as compared to walking without the suit. They mentioned about feeling of assistive force while walking but are not sure how it reduces the efforts. They also mentioned that the backpack is lightweight and does not feel load on the back, but most farmers need to visit paddy fields using small trucks. In such cases, wearing and removing the suit due to backpack is time consuming. In our evaluation experiment we observed AWGAS reduces lower limb’s muscle efforts while walking. The feedback received during trials complemented the evaluated results. This proves that we can use AWGAS not in controlled environment but also in the uncontrolled environment. During trial we identified two areas of improvements. First making compact design will improve usability for their daily lifestyle. Second, the current system uses air tank which lasts for 500 actuations as described in section 2.2. During the trial we used two air tanks so that it lasts longer. We feel, a better management of air tanks or strategy to replace the tanks is needed to support large number of actuations. 


\begin{figure}[h]
	\centering
	\includegraphics[width=0.9\linewidth]{photos/preliminarystudy}
	\caption{Preliminary trial in kitahiroshima, Hiroshima, Japan. Photo shows elderly person testing AWGAS in the farm land}
	\label{fig:preliminarystudy}
\end{figure}

\section{Discussion}
In this study, we developed the AWGAS, a lightweight wearable walking assist suit, a PGM actuation control mechanism for assisting forward locomotion during the swing phase of the gait cycle, and the gait cycle detection system. Due to the use of portable air tanks the AWGAS overcomes the limitation of UPS by supporting variable walking speed and no actuation delay. The current configuration provides assistive force from the pre-swing to the beginning stance phase of the gait cycle, which constitutes 40\% to 50\% of a gait cycle. From the results of our experiment, we observed that the AWGAS could reduce muscle efforts of the quadriceps femoris and hamstring muscles along which the PGM was attached, and reduced the calf muscles and tibialis anterior efforts, which are primarily active during the swing phase of the gait cycle. This suggests that the assisting forward locomotion of the gait cycle augments human walking by reducing muscle activation during walking. Based on assistive force model discussed in Eq. \ref{eq:fpgm} PGM can exert force even in the absence of air pressure supply. Experimental evaluation to support this claim shows, when wearing AWGAS in the absence of air pressure, it reduces muscle activation during walking significantly as shown in \ref{fig:awgasemgbarpvalue}. Such ability of the wearable walking assist is developed in very few exoskeletons such as \cite{23} where the assistance is discussed based on the reduction metabolic cost of walk while using unpowered exoskeleton. 

As discussed earlier in introduction section existing devices using PAM solves their purpose in their own unique methods and device control techniques. N. Aliman et al. in his survey of lower limb exoskeleton identified various exoskeleton devices developed using PAM \cite{24}. But these devices face challenge of ease of use and portability due to involvement of rigid structure and weight of the device. Large powered sources such as compressor or air tank makes device heavy and not portable which limits its usage in controlled environment such as factories or hospitals. In our research we developed AWGAS using PGM to overcome some of these challenges. AWGAS weights 2 kg including portable air tanks, suit uses fabric material to attach PGM using Velcro tapes. This configuration distinguishes AWGAS from existing devices where no rigid material is used, the device is soft and portable, easy to wear and augments walking even in the absence of powered source due to force generating characteristic of PGM. 

The PGM used in AWGAS can work with low air pressure but has lower force generating capacity than traditional PAM. Due to this AWGAS cannot be directly used for rehabilitation where larger assistive force is needed. We believe devices like \cite{5,8} are good use of pneumatic actuators for industrial and rehabilitation and healthcare assistive device due to large force carrying capacity. Whereas AWGAS and devices like these are useful for augmenting human walking or other motion by reducing required muscle efforts.




\section{Conclusion}

In this paper, we discussed design and development of AWGAS (assistive wearable gait augmenting suit), swing phase detection system and PGM actuation control. The system is realized as feedforward control where assistive force is provided when the swing phase is detected. The configuration provides assistive force for 40\% to 50\% of the gait cycle. The experiment shows a reduction in the lower limb muscle activation significantly, and no adverse effect regarding excessive muscle activation. AWGAS can assist variable walking speed which is possible due to the implementation of gait detection system and use of portable air tank. AWGAS is lightweight, portable, easy to use walking assist suit and can assist walking even in the absence of supplied air pressure. Further, in our study, we plan to develop variable stiffness modulation control of PGM, it can provide assistive force based on actual requirement. Such control will be useful for improving lower limb locomotion and assistance for various other tasks involved bending hip or knee for pick and place operations. The current configuration uses FSR sensors in the shoe to detect the stance and swing phases of gait cycle, we believe inertial sensors along with FSR can be used to designed more detailed control strategy for dynamic requirements of human walking. In addition to this we also plan to add support for additional PGMs if required without change in controller and gain addition strength, assistive force or replacing the damaged artificial muscles easily. 

\acknowledgements
The authors take this opportunity to thank members of Biological Systems Engineering lab at Graduate School of Engineering in Hiroshima University, Japan for participating in the performance evaluation of the AWGAS. We also like to thanks Daiya Industries for development and support of low pressure driven pneumatic gel muscle i.e. PGM.

This research is supported by JST PRESTO Grant Number JPMJPR16D3.

The author Chetan Thakur was supported through the Hiroshima University
TAOYAKA Program for creating a flexible, enduring, peaceful society, funded by the Program for Leading
Graduate Schools, Ministry of Education, Culture, Sports, Science and Technology.  


%\bibliographystyle{unsrt}%if you use bibtex
%\bibliography{template}

%[IMPORTANT]
%To clarify research positioning and purpose, authors should survey international literatures, 
%including JRM publications, and list them in references. It is strongly recommended that 
%authors include JRM publications in references (JRM is under evaluation of SCI to get IF). 
%All papers appear in the JRM and PDFs made available free of charge at the following website 
%[OPEN ACCESS] at 
%https://www.fujipress.jp/jrm/rb/
%Create your account for download for free at 
%https://www.fujipress.jp/usces-member/?page=newmember


\begin{thebibliography}{99}

\bibitem{1}	K. Suzuki, G. Mito, H. Kawamoto, Y. Hasegawa, and Y. Sankai, “Intention-Based Walking Support for Paraplegia Patients with Robot Suit HAL,” Adv. Robot., vol. 21, no. 12, pp. 1441–1469, 2007.
\bibitem{2}	S. Toyama and G. Yamamoto, “Development of wearable-agri-robot - Mechanism for agricultural work,” in 2009 IEEE/RSJ International Conference on Intelligent Robots and Systems, IROS 2009, 2009, pp. 5801–5806.
\bibitem{3}	Y. Ikeuchi, J. Ashihara, Y. Hiki, H. Kudoh, and T. Noda, “Walking assist device with bodyweight support system,” in 2009 IEEE/RSJ International Conference on Intelligent Robots and Systems, IROS 2009, 2009, pp. 4073–4079.
\bibitem{4}	P. Malcolm, W. Derave, S. Galle, and D. De Clercq, “A Simple Exoskeleton That Assists Plantarflexion Can Reduce the Metabolic Cost of Human Walking,” PLoS One, vol. 8, no. 2, 2013.
\bibitem{5}	M. Ishii, K. Yamamoto, and K. Hyodo, “Stand-Alone Wearable Power Assist Suit - Development and Availability,” J. Robot. Mechatronics, vol. 17, no. 5, pp. 17–18, 2005.
\bibitem{6}	Y. Naruoka, N. Hiramitsu, and Y. Mitsuya, “A study of power-assist technology to reduce body burden during loading and unloading operations by support of knee joint motion,” J. Robot. Mechatronics, vol. 28, no. 6, pp. 949–957, 2016.
\bibitem{7}	K. Sano, E. Yagi, and M. Sato, “Development of a Wearable Assist Suit for Walking and Lifting-Up Motion Using Electric Motors,” J. Robot. Mechatronics, vol. 25, no. 6, pp. 923–930, 2013.
\bibitem{8}	T. Noritsugu, D. Sasaki, M. Kameda, A. Fukunaga, and M. Takaiwa, “Wearable power assist device for standing up motion using pneumatic rubber artificial muscles,” J. Robot. Mechatronics, vol. 19, no. 6, pp. 619–628, 2007
\bibitem{9}	B. T. Quinlivan et al., “Assistance magnitude versus metabolic cost reductions for a tethered multiarticular soft exosuit,” vol. 4416, no. January, 2017
\bibitem{10}	A. T. Asbeck, K. Schmidt, and C. J. Walsh, “Soft exosuit for hip assistance,” Rob. Auton. Syst., vol. 73, pp. 102–110, 2015.
\bibitem{11}	K. Schmidt et al., “The myosuit: Bi-articular anti-gravity exosuit that reduces hip extensor activity in sitting transfers,” Front. Neurorobot., vol. 11, no. OCT, pp. 1–16, 2017.
\bibitem{12}	A. T. Asbeck, S. M. M. De Rossi, K. G. Holt, and C. J. Walsh, “A biologically inspired soft exosuit for walking assistance,” Int. J. Rob. Res., vol. 34, no. 6, pp. 744–762, 2015.
\bibitem{13}	K. Ogawa, C. Thakur, T. Ikeda, T. Tsuji, and Y. Kurita, “Development of a pneumatic artificial muscle driven by low pressure and its application to the unplugged powered suit,” Adv. Robot., vol. 31, no. 21, pp. 1135–1143, 2017.
\bibitem{14}	D. Daerden Frank; Lefeber, “Pneumatic artificial muscles: actuators for robotics and automation,” Eur J Mech Eng, vol. 47, no. 1, pp. 10–21, 2000.
\bibitem{15}	J. Perry and J. Burnfield, “GAIT Normal and Pathological Function,” J. Sports Sci. Med., vol. 9, no. 2, p. 551, Jun. 2010.
\bibitem{16}	S. Chen, J. Lach, B. Lo, and G. Z. Yang, “Toward Pervasive Gait Analysis With Wearable Sensors: A Systematic Review,” IEEE J. Biomed. Heal. Informatics, vol. 20, no. 6, pp. 1521–1537, 2016.
\bibitem{17}	Y. Long, Z. jiang Du, W. dong Wang, and W. Dong, “Human motion intent learning based motion assistance control for a wearable exoskeleton,” Robot. Comput. Integr. Manuf., vol. 49, no. July 2017, pp. 317–327, 2018.
\bibitem{18}	J. Taborri, E. Palermo, S. Rossi, and P. Cappa, “Gait partitioning methods: A systematic review,” Sensors (Switzerland), vol. 16, no. 1, pp. 40–42, 2016.
\bibitem{19}	C. Kirtley, M. W. Whittle, and R. J. Jefferson, “Influence of walking speed on gait parameters,” J. Biomed. Eng., vol. 7, no. 4, pp. 282–288, 1985.
\bibitem{20}	T. P. Andriacchi, J. A. Ogle, and J. O. Galante, “Walking speed as a basis for normal and abnormal gait measurements,” J. Biomech., vol. 10, no. 4, pp. 261–268, 1977.
\bibitem{21}	S. Oh, E. Baek, S. K. Song, S. Mohammed, D. Jeon, and K. Kong, “A generalized control framework of assistive controllers and its application to lower limb exoskeletons,” Rob. Auton. Syst., vol. 73, pp. 68–77, 2015.
\bibitem{22}	R. W. Kressig and O. Beauchet, “Guidelines for clinical applications of spatio-temporal gait analysis in older adults,” Aging Clin. Exp. Res., vol. 18, no. 2, pp. 174–176, Apr. 2006.
\bibitem{23}	S. H. Collins, M. B. Wiggin, and G. S. Sawicki, “Reducing the energy cost of human walking using an unpowered exoskeleton,” Nature, vol. 522, no. 7555, pp. 212–215, 2015.
\bibitem{24}	N. Aliman, R. Ramli, and S. M. M. Haris, “Design and development of lower limb exoskeletons: A survey,” Rob. Auton. Syst., vol. 95, pp. 102–116, 2017.


\end{thebibliography}

\end{document}
